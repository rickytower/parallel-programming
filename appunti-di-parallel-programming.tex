\documentclass[italian,10pt,a4paper]{article}
\usepackage[T1]{fontenc}
\usepackage{graphicx}
\usepackage{mathtools}
\usepackage{amssymb}
\usepackage{amsthm}
\usepackage{thmtools}
\usepackage{xcolor}
\usepackage{nameref}
\usepackage{longtable}
\usepackage{tabularray}
\usepackage{tabularx}
\usepackage{microtype}
\usepackage{babel}
\usepackage[hidelinks]{hyperref}
\usepackage{mdframed}
{\renewcommand{\arraystretch}{2}%
\title{Alcuni concetti utilizzati negli esercizi di parallel programming}
\author{Riccardo Torre}
\newtheorem{exercise}{Esercizio}[section]
\newtheorem{solution}{Soluzione}[section]
\begin{document}
	\maketitle
	\tableofcontents
	\section{Introduzione}
Un programma eseguito su un'architettura dotata di una singola unità di elaborazione centrale (CPU) può essere eseguita esclusivamente in maniera sequenziale (un'istruzione alla volta) in figura \ref{fig:single-computation}
% TODO: \usepackage{graphicx} required
\begin{figure}[th]
	\centering
	\includegraphics[width=0.7\linewidth]{img/single-computation}
	\caption{esecuzione su un'architettura dotata di una sola CPU.}
	\label{fig:single-computation}
\end{figure}

	\section{Terminologia e concetti}
L'architettura di Von Neumann è composta da quattro componenti principali:
\begin{enumerate}
	\item memoria;
	\item unità di controllo (CU): recupera istruzioni/dati dalla memoria, decodifica le istruzioni e poi in \textbf{sequenza} coordina le operazioni per portare a termine il compito programmato;
	\item unità logico aritmetica (ALU): esegue operazioni aritmetiche di base;
	\item sistemi di I/O (input/output): interfaccia uomo-macchina.
\end{enumerate}
La \textbf{tassonomia classica di Flynn} distingue le architetture di computer multiprocessore in base a come possono essere classificate in due dimensioni indipendenti di \textbf{istruzioni} e \textbf{dati}.
Nella tassonomia di Flynn, vengono identificate cinque componenti:
\begin{enumerate}
	\item IS (Instruction Stream): ovvero il flusso di istruzioni del programma che deve essere eseguito;
	\item DS (Data Stream): il flusso degli operandi e dei risultati dei programmi che sono in esecuzione;
	\item CU (Control Unit): elemento funzionale (chi esegue il fetch e il decode, ovvero il recupero e la decodifica dei dati/istruzioni);
	\item PU (Processing Unit): unità funzionale composta dalla ALU e dai registri. Esegue le istruzioni;
	\item MM(Main Memory): la memoria dove i dati e le istruzioni sono allocati.
\end{enumerate}
Le possibili classificazioni sono:
\begin{enumerate}
	\item SISD (Single Instruction Single Data): la CU recupera le istruzioni dalla MM, mentre la PU esegue le istruzioni interagendo con MM per modificare i dati (figura \ref{fig:sisd});
	\begin{figure}[th]
		\centering
		\includegraphics[width=0.7\linewidth]{img/sisd}
		\caption{Single Istruction Single Data.}
		\label{fig:sisd}
	\end{figure}
	\item SIMD (Single Instruction Multiple Data): tutte le unità di elaborazione eseguono la stessa istruzione in ogni dato ciclo di clock; ogni unità di elaborazione può operare su un dato diverso (figura \ref{fig:simd}); è ideale per problemi specializzati caratterizzati da un elevato grado di regolarità, come l'elaborazione grafica o l'elaborazione di immagini; esecuzione sincrona (\textbf{lockstep}) e deterministica
	\begin{figure}[th]
		\centering
		\includegraphics[width=0.7\linewidth]{img/simd}
		\caption{Single Istruction Multiple Data.}
		\label{fig:simd}
	\end{figure}

	\item MISD (Multple Instruction Single Data): un singolo flusso di dati viene immesso in più unità di elaborazione che opera sui dati in modo indipendente tramite flussi di istruzioni indipendenti. Sono esistiti pochi esempi reali di questa classe di computer paralleli (figura \ref{fig:misd});
	\begin{figure}[th]
		\centering
		\includegraphics[width=0.7\linewidth]{img/misd}
		\caption{Multiple Instruction Single Data.}
		\label{fig:misd}
	\end{figure}

	\item MIMD (Multple Instruction Multple Data): è il tipo più comune di computer parallelo; ogni processore può eseguire un flusso di istruzioni diverso e lavorare con un flusso di dati diverso. L'esecuzione può essere sincrona o asincrona, deterministica o non deterministica (figura \ref{fig:mimd}).
	\begin{figure}[!th]
		\centering
		\includegraphics[width=0.7\linewidth]{img/mimd}
		\caption{Multiple Instruction Multiple Data.}
		\label{fig:mimd}
	\end{figure}

\end{enumerate}
\subsection{Terminologia delle architetture parallele}

	\begin{longtable}{|m{0.28\linewidth}|m{0.62\linewidth}|}
		\hline
		\textbf{Termine} & \textbf{Significato}
		\\
		\hline
		\textbf{Task (sequenziale)} & È un'unità di esecuzione o di lavoro di un programma o di un sottoprogramma. Le istruzioni sono eseguite sequenzialmente ovvero non in parallelo.
		\\
		\hline
		\textbf{Task parallelo}& È un task che può essere eseguito in modo sicuro su più processori. Ciascuna istruzione di un task può essere eseguita in maniera concorrente.
		\\
		\hline
		\textbf{Esecuzione seriale} & È l'esecuzione di un programma in sequenza, una dichiarazione alla volta. Nel senso più semplice questo accade su una macchina con un processore. Tuttavia, praticamente tutte le attività parallele avranno sezioni di un programma parallelo che devono essere eseguite in serie.
		\\
		\hline
		\textbf{Esecuzione parallela} & È l'esecuzione di un programma da più di un task, in cui ogni attività è in grado di eseguire la stessa o una diversa istruzione simultaneamente.
		\\
		\hline
		\textbf{Pipelining} & Suddivisione di un compito in passaggi eseguiti da diverse unità di elaborazione, con input che scorrono proprio come una catena di montaggio; un tipo di calcolo parallelo.  La pipeline è un tipo di architettura hardware utilizzata nel progetto dei microprocessori per incrementarne la produttività, ovvero la quantità di istruzioni eseguite nell'unità di tempo. Le istruzioni possono essere di un programma sequenziale. La pipeline può riferirsi anche ad una pipeline di task (ottimizzazione per GPU).
		\\
		\hline
		\textbf{Memoria condivisa} & Dal punto di vista strettamente hardware, descrive un'architettura informatica alla quale tutti i processori hanno accesso diretto (solitamente basato su bus) alla memoria fisica comune; dal punto di vista software, descrive un modello in cui le attività parallele hanno tutte la stessa "immagine" di memoria (\textbf{spazio di indirizzamento comune}) e possono indirizzarsi e accedervi direttamente alle stesse locazioni di memoria logiche indipendentemente da dove esiste effettivamente la memoria fisica. Se lo spazio di indirizzamento non fosse condiviso, si parla di \textbf{memoria distribuita} (che è il contrario della memoria condivisa).
		\\
		\hline
		\textbf{Multiprocessore simmetrico (SMP)} & Architettura hardware in cui più processori condividono un unico spazio di indirizzi e accedono a tutte le risorse; la computazione avviene su memoria condivisa.
		\\
		\hline
		\textbf{Memoria distribuita} & Dal punto di vista hardware, si riferisce all'accesso alla memoria basato sulla rete per la memoria fisica che non è in comune. Il termine può fare riferimento anche ad un'implementazione software dove le attività possono vedere solo la memoria della macchina locale su una vista logica e devono utilizzare comunicazioni per accedere alla memoria su altre macchine su cui sono in esecuzione altre attività.
		\\
		\hline
		\textbf{Comunicazione} & Le attività parallele in genere necessitano di scambiare dati. Esistono tuttavia diversi modi per ottenere questo risultato, ad esempio tramite un bus di memoria condiviso o tramite una rete, tuttavia l'evento effettivo dello scambio di dati viene comunemente definita \emph{comunicazione} indipendentemente dal metodo utilizzato.
		\\
		\hline
		\textbf{Sincronizzazione} & È il coordinamento di attività parallele in tempo reale, molto spesso associato a comunicazioni. Spesso implementate da stabilire un punto di sincronizzazione all'interno di un'applicazione in cui un'attività non può procedere oltre finché un'altra attività non raggiunge lo stesso punto o logicamente equivalente. La sincronizzazione di solito comporta l'attesa di almeno un'altra attività e può quindi causare un aumento del tempo di esecuzione del clock delle applicazioni parallele.
		\\
		\hline
		\textbf{Granularità} & Nel calcolo parallelo, la granularità è una misura quantitativa del rapporto tra il calcolo e la comunicazione. La granularità può essere:
		\begin{itemize}
			\item \textbf{grossolana:} tra gli eventi di comunicazione vengono eseguite quantità di lavoro computazionale relativamente grandi;
			\item \textbf{fine:} tra gli eventi di comunicazione vengono eseguite quantità di lavoro computazionale relativamente piccole.
		\end{itemize}
		\\
		\hline
		\textbf{Accelerazione osservata} & È definita come: $speedup = \frac{execTimeSerial}{execTimeParallel}$ e rappresenta uno degli indicatori più semplici e utilizzati per la performance di un programma parallelo.
		\\
		\hline
		\textbf{Overhead parallelo} & È l'ammontare di tempo richiesto per coordinare task paralleli anziché per eseguire la parte utile del lavoro. L'overhead parallelo può includere fattori come:
		\begin{itemize}
			\item tempo di avvio dei task;
			\item tempo per le sincronizzazioni tra task;
			\item tempo per le comunicazioni di dati tra task;
			\item overhead del software imposto da compilatori paralleli, librerie, strumenti, sistema operativo, \dots
			\item tempo di completamento dei task.
		\end{itemize}
		\\
		\hline
		\textbf{Massivamente parallelo} &  Si riferisce all'hardware che comprende un dato sistema parallelo, dotato di molti processori. Il significato di "molti" continua ad aumentare, ma attualmente i computer più grandi possono essere costituiti da processori che si contano nell'ordine di centinaia di migliaia.
		\\
		\hline
		\textbf{Imbarazzantemente parallelo} & Indica l'azione di risolvere simultaneamente task simili ma indipendenti; la coordinazione tra task è quasi assente o completamente assente.
		\\
		\hline
		\textbf{Scalabilità} & Si riferisce alla capacità di un sistema parallelo (HW e/o SW) di dimostrare un aumento proporzionale della velocità parallela con l'aggiunta di più processori. I fattori che contribuiscono alla scalabilità includono:
		\begin{itemize}
			\item HW: in particolare larghezze di banda della memoria-cpu e comunicazioni attraverso la rete;
			\item l'algoritmo dell'applicazione;
			\item relativo all'overhead parallelo;
			\item caratteristiche specifiche dell'applicazione e codifica
		\end{itemize}
		\\
		\hline
		\textbf{Processori multi-core} & Processori in cui sono presenti più core su un singolo chip
		\\
		\hline
		\textbf{Computazione in cluster} & Utilizzo di una combinazione di unità di calcolo di base (processori, reti o SMP) per costruire un sistema parallelo.
		\\
		\hline
		\textbf{Supercomputing / Calcolo ad alte prestazioni} & Utilizzo delle macchine più veloci e più grandi al mondo per risolvere problemi molto complessi (le unità di calcolo di cui sono composti non devono necessariamente essere omogenei).
		\\
		\hline
		\textbf{Edge computing} & Distribuire il paradigma informatico che avvicina il calcolo e l'archiviazione dei dati al luogo in cui si trovano. Necessario per migliorare i tempi di risposta e risparmiare larghezza di banda.
		\\
		\hline
	\end{longtable}
\subsubsection{Tipi di memoria}
La \textbf{cache} è un'area di memoria estremamente veloce ma solitamente di un basso ordine di grandezza di capacità. Il suo scopo è di velocizzare l'esecuzione dei programmi.

La figura \ref{fig:uma} è un esempio di memoria condivisa. I computer paralleli con memoria condivisa variano ampiamente, ma generalmente hanno in comune la capacità di tutti i processori di accedere a tutta la memoria come spazio di indirizzi globale. Più processori possono funzionare in modo indipendente ma condividono le stesse risorse di memoria. Le modifiche in una posizione di memoria effettuate da un processore sono visibili a tutti gli altri processori. Le macchine a memoria condivisa possono essere divise in due classi principali in base ai tempi di accesso alla memoria:
\begin{enumerate}
	\item UMA (Unified Memory Access): i processori sono identici, il tempo e la modalità di accesso alla memoria sono gli stessi per tutti i processori (figura \ref{fig:uma}). A volte chiamato \textbf{CC-UMA} (\emph{Cache Coherent UMA}). \textbf{Coerenza} della cache significa che se un processore aggiorna una posizione nella memoria condivisa, tutti gli altri processori vengono a conoscenza dell'aggiornamento. La coerenza della cache viene raggiunta a livello hardware.
	\begin{figure}[th]
		\centering
		\includegraphics[width=0.7\linewidth]{img/UMA}
		\caption{Memoria condivisa di classe UMA.}
		\label{fig:uma}
	\end{figure}
	\item NUMA (Not Unified Memory Access): spesso realizzato collegando fisicamente due o più SMP; un SMP può accedere direttamente alla memoria di un altro SMP. Non tutti i processori hanno lo stesso tempo di accesso a tutte le memorie. L'accesso alla memoria attraverso il bus è più lento. Se viene mantenuta la coerenza della cache, la memoria si dice \textbf{CC-NUMA} (\emph{Cache Coherent NUMA}).
	\begin{figure}[th]
		\centering
		\includegraphics[width=0.7\linewidth]{img/NUMA}
		\caption{Memoria condivisa di classe NUMA.}
		\label{fig:numa}
	\end{figure}
\end{enumerate}
	\subsubsection{Vantaggi e svantaggi di una memoria condivisa.} Vantaggi di avere una memoria condivisa sono:
	\begin{itemize}
		\item lo spazio degli indirizzi globale fornisce alla memoria una prospettiva di programmazione user-friendly; \item la condivisione dei dati tra le attività è veloce e uniforme alla vicinanza della memoria alle CPU.
	\end{itemize}
	Gli svantaggi sono:
	\begin{itemize}
		\item mancanza di scalabilità tra memoria e CPU; \item diventa sempre più difficile e costoso progettare e produrre macchine a memoria condivisa con un numero sempre crescente di processori.
	\end{itemize}

Come i sistemi di memoria condivisa, i sistemi di memoria distribuita variano ampiamente ma condividono una caratteristica comune. Per connettere la memoria inter-processore, il sistema richiede una rete di comunicazione.
I processori hanno la propria memoria locale. Gli indirizzi di memoria in un processore non vengono mappati su un altro processore, quindi non esiste il concetto di \textbf{spazio degli indirizi globale} su tutti i processori.

Poiché ogni processore ha la propria memoria locale, funziona in modo indipendente. Le modifiche apportate alla memoria non hanno effetto sulla memoria degli altri processori. Pertanto, il concetto di coerenza della cache non si applica.
Quando un processore ha bisogno di accedere ai dati in un altro processore, solitamente è compito del programmatore definire esplicitamente come e quando i dati vengono comunicati. Anche la sincronizzazione tra i compiti è responsabilità del programmatore.

\subsubsection{Vantaggi e svantaggi di una memoria distribuita.} I vantaggi di avere una memoria distribuita sono:
\begin{itemize}
	\item la \textbf{memoria} è \textbf{scalabile} con il numero di processori; \item aumenta il numero di processori e la dimensione della memoria aumenta proporzionalmente; \item ogni processore può accedere rapidamente alla propria memoria senza interferenze e senza il sovraccarico sostenuto nel tentativo di mantenere la coerenza della cache. \item in termini di costi, la memoria distribuita è efficace, poiché può utilizzare processori e reti standard disponibili in commercio.
\end{itemize}
Gli svantaggi sono i seguenti:
\begin{itemize}
	\item il programmatore è responsabile per molti dei dettagli associati alla comunicazione dei dati tra i processori. \item potrebbe essere difficile mappare su questa organizzazione della memoria le strutture dati esistenti, basate sulla memoria globale; \item tempi di accesso alla memoria non uniforme (NUMA).
\end{itemize}
\subsubsection{Memoria ibrida distribuita-condivisa} I computer più grandi e veloci del  mondo oggi impiegano questo tipo di architettura (figura \ref{fig:hybrid-memory}). Il componente di memoria condivisa è solitamente una macchina SMP coerente con la cache. I processori su un dato SMP possono indirizzare la memoria di quella macchina come globale.

\begin{figure}[th]
	\centering
	\includegraphics[width=0.7\linewidth]{img/hybrid-memory}
	\caption{Memoria ibrida distribuita-condivisa.}
	\label{fig:hybrid-memory}
\end{figure}

Il componente di memoria distribuita è il collegamento in rete di più SMP. Gli SMP conoscono solo la propria memoria, non quella di un altro SMP. Pertanto, le comunicazioni di rete sono necessarie per spostare i dati da un SMP all'altro. I vantaggi e gli svantaggi sono tutto ciò che è comune alle architetture di memoria e distribuita.

	\section{Modelli di programmazione parallela}
Esistono diversi modelli di programmazione parallela comunemente usati come la \textbf{memoria condivisa}, le \textbf{threads}, il \textbf{passaggio di messaggi}, il \textbf{parallelismo dei dati} e un approccio \textbf{ibrido}. Questi modelli sono un astrazione dell'hardware e delle architetture di memoria sottostanti.

Questi modelli non sono specifici di una particolare macchina o architettura di memoria. In effetti, ciascuno di questi modelli può essere implementato su qualsiasi hardware sottostante. Due esempi:
\begin{enumerate}
	\item modello di memoria condivisa su una memoria distribuita: approccio KSR (\textit{Kendall Square Research}) ALLCACHE. La memoria della macchina era distribuita fisicamente, ma appariva all'utente come un'unica memoria condivisa (spazio di indirizzi globale). Genericamente questo approccio viene definito come \textit{"macchina virtuale condivisa"};
	\item modello di passaggio del messaggio su una macchina con memoria condivisa. MPI su SGI Origin. SGI Origin utilizzava un'architettura di memoria condivisa di tipo CC-NUMA, in cui ogni attività ha accesso diretto alla memoria globale. Tuttavia, la capacità di inviare e ricevere messaggi con MPI, come avviene comunemente su una rete di macchine a memoria distribuita, non solo è implementata ma è utilizzata molto comunemente.
\end{enumerate}

Quale modello utilizzare è spesso una combinazione di ciò che è disponibile e di una scelta personale. Non esiste un modello "migliore", anche se esistono certamente implementazioni migliori di alcuni modelli rispetto ad altri.

\subsection{Modello di memoria condivisa} Nel modello di programmazione a memoria condivisa, le attività condividono uno spazio di indirizzi comune, che leggono e scrivono in modo asincrono. Vari meccanismi come lock o semafori possono essere utilizzati per controllare l'accesso alla memoria condivisa. Un vantaggio di questo modello dal punto di vista del programmatore è che manca la nozione di "proprietà" dei dati. Infatti non è necessario specificare esplicitamente la comunicazione dei dati tra le attività. Lo sviluppo del programma può essere semplificato. Uno svantaggio importante in termini di prestazioni è che diventa difficile comprendere e gestire la località dei dati. In effetti, risulta necessario mantenere i dati locali rispetto al processore che li utilizza, preserva gli accessi alla memoria, gli aggiornamenti della cache e il traffico del bus che si verifica quando più processori utilizzano gli stessi dati.

\subsubsection{Implementazione del modello di memoria condivisa} Sulle piattaforme di memoria condivisa, i compilatori nativi traducono variabili del programma utente in indirizzi di memoria effettivi, che sono globali. Attualmente non esistono implementazioni comuni della piattaforma di memoria distribuita. Tuttavia, l'approccio KSR ALLCACHE forniva una visualizzazione della memoria condivisa dei dati anche se la memoria fisica della macchina era distribuita.
\subsection{Modello con le thread}
\begin{figure}[th]
	\centering
	\includegraphics[width=0.7\linewidth]{img/modello-thread}
	\caption{modello con le thread.}
	\label{fig:modello-thread}
\end{figure}
Nel modello con le thread della programmazione parallela (figura \ref{fig:modello-thread}), un singolo processo può avere più percorsi di esecuzione simultanei. Forse l'analogia più semplice che può essere utilizzata per descrivere le thread, è il concetto di un singolo programma che include un numero di subroutine.
\begin{itemize}
	\item il programma principale a.out è pianificato per l'esecuzione dal sistema operativo. a.out carica ed acquisisce tutte le risorse utente e di sistema necessarie per l'esecuzione;
	\item a.out esegue del lavoro seriale e quindi crea una serie di thread che possono essere pianificate ed eseguite contemporaneamente dal sistema operativo;
	\item ogni thread ha dati locali, ma condivide anche tutte le risorse di a.out. Ciò consente di risparmiare il sovraccarico associato alla replica delle risorse di un programma per ciascuna thread. Ogni thread beneficia anche di una vista globale della memoria perché condivide lo spazio di memoria di a.out;
	\item il lavoro di una thread può essere meglio descritto come una subroutine all'interno del programma principale. Qualsiasi thread può eseguire qualsiasi subroutine contemporaneamente agli altri thread.
	\item le thread comunicano tra loro attraverso la memoria globale (in cui gli indirizzi vengono aggiornati). Ciò richiede costrutti di sincronizzazione per garantire che più di un thread non aggiorni lo stesso indirizzo globale in qualsiasi momento;
	\item le thread possono nascere e morire, ma a.out rimane presente per fornire le risorse condivise necessarie fino al completamento dell'applicazione.
\end{itemize}

\subsubsection{Implementazione del modello dei thread} Le thread sono comunemente associate ad architetture di memoria condivisa e a sistemi operativi. Dal punto di vista della programmazione, le implementazioni delle thread comprendono comunemente:
\begin{itemize}
	\item una libreria di subroutine chiamate dal codice sorgente parallelo \item un insieme di direttive del compilatore incorporate nel codice sorgente seriale o parallelo.
\end{itemize}
In entrambi i casi, il programmatore è responsabile della determinazione di tutto il parallelismo.

Le implementazioni threaded non sono nuove nell'informatica. Storicamente, i fornitori di hardware hanno implementate le proprie versioni proprietarie delle thread, che differivano sostanzialmente l'una dall'altra, rendendo difficile per i programmatori sviluppare applicazioni threaded portatili.

Gli sforzi di standardizzazione non correlati hanno portato a due implementazioni molto diverse delle thread:
\begin{enumerate}
	\item thread POSIX: sono specifiche del linguaggio C; sono comunemente indicate come \textit{Pthreads}; molti venditori di hardware le forniscono in aggiunta alle loro implementazioni proprietarie; il parallelismo è molto esplicito;
	\item OpenMP: è basato sulle direttive del compilatore; può utilizzare il codice seriale; definito e approvato congiuntamente da un gruppo di importanti fornitori di hardware e software per computer; è disponibile nelle implementazioni C/C++ e Fortran; è molto facile e semplice da usare: prevede il "parallelismo incrementale".
\end{enumerate}
\subsection{Modello di trasmissione dei messaggi}
\begin{figure}[th]
	\centering
	\includegraphics[width=0.7\linewidth]{img/modello-trasmissione-messaggi}
	\caption{modello di trasmissione dei messaggi.}
	\label{fig:modello-trasmissione-messaggi}
\end{figure}
Il modello di trasmissione dei messaggi (figura \ref{fig:modello-trasmissione-messaggi}) dimostra le seguenti caratteristiche:
\begin{itemize}
	\item una serie di task che usano la propria memoria locale durante la computazione. Tasks multipli possono risiedere fisicamente sulla stessa macchina e su un numero arbitrario di macchine;
	\item i task comunicano attraverso l'invio e la ricezione di messaggi;
	\item il trasferimento dei messaggi di solito richiede l'esecuzione di operazioni cooperative da parte di ciascun processo. Ad esempio, un'operazione di invio deve avere un'operazione di ricezione corrispondente.
\end{itemize}

\subsubsection{Implementazione del modello di passaggio dei messaggi} Dal punto di vista della programmazione, le implementazioni dello scambio di messaggi comprendono comunemente una libreria di subroutine incorporate nel codice sorgente. Il programmatore è responsabile di tutto il parallelismo. Nel 1992 è stato formato il Forum MPI con l'obbiettivo primario di stabilire un'interfaccia standard per le implementazioni dello scambio di messaggi. MPI è ora lo standard industriale "de facto" per lo scambio di messaggi.

Per le architetture a memoria condivisa, le implementazioni MPI solitamente non utilizzano una rete per le comunicazioni delle attività. Usano la propria memoria condivisa per motivi di prestazioni.

\subsection{Modello data parallel}
\begin{figure}[th]
	\centering
	\includegraphics[width=0.7\linewidth]{img/modello-parallelo-dei-dati}
	\caption{modello data parallel.}
	\label{fig:modello-parallelo-dei-dati}
\end{figure}
Il modello data parallel (figura \ref{fig:modello-parallelo-dei-dati}) ha le seguenti caratteristiche:
\begin{itemize}
	\item la maggior parte del lavoro si concentra sull'esecuzione di operazioni su un set di dati. Il set è generalmente organizzato in una \textbf{struttura  dati comune}, ad esempio un array o un cubo;
	\item una serie di attività lavora collettivamente sulla stessa struttura dati, tuttavia, ogni attività funziona su una partizione diversa della stessa struttura dati;
	\item le attività eseguono la stessa operazione sulla loro partizione di lavoro, ad esempio "aggiungi 4 a ogni elemento dell'array";
	\item nelle architetture a memoria condivisa, tutte le attività possono avere accesso alla struttura dei dati attraverso la memoria globale. Nelle architetture di memoria distribuita la struttura dei dati è suddivisa e risiede come "chunks" (ovvero "pezzi") nella memoria locale di ciascun task.
\end{itemize}

\subsubsection{Implementazione del modello data parallel} La programmazione con il modello data parallel viene solitamente compiuta scrivendo un programma con costrutti data parallel. I costrutti possono essere chiamati tramite subroutine di una libreria che supporta data parallel o attraverso direttive per un compilatore data parallel.

Le implementazioni più comuni sono:
\begin{itemize}
	\item HPF (High Performance Fortran);
	\item direttive di compilatore;
\end{itemize}

\subsection{Modello ibrido} È un modello in cui possono essere combinati due o più modelli di programmazione parallela. Un esempio comune è di combinare il MPI (Message Passing Model) con il modello delle thread (thread POSIX) o in alternativa il modello della memoria condivisa (OpenMP).

Un altro esempio comune di modello ibrido consiste nel combinare i dati in parallelo con lo scambio di messaggi.
\subsection{Modello SPMD (Single Program Multiple Data)} SPMD è in realtà un modello di programmazione di "alto livello" che può essere costruito su qualsiasi combinazione dei modelli di programmazione parallela precedentemente menzionati. Un singolo programma viene eseguito da tutti i task contemporaneamente. In qualsiasi momento, le attività possono eseguire le stesse o differenti istruzioni all'interno dello stesso programma (figura \ref{fig:spmd}).
\begin{figure}[th]
	\centering
	\includegraphics[width=0.7\linewidth]{img/SPMD}
	\caption{modello SPMD.}
	\label{fig:spmd}
\end{figure}
A differenza di SIMD (Single Instruction Multiple Data), nel SPMD, più processori autonomi eseguono simultaneamente lo stesso programma in punti indipendenti, anziché in unico \textbf{lockstep} che SIMD impone su dati diversi.\footnote{I sistemi lockstep sono sistemi informatici tolleranti agli errori che eseguono lo stesso insieme di operazioni contemporaneamente in parallelo.} Con SPMD, le attività possono essere eseguite per CPU general purpose; SIMD richiede processori vettoriali per manipolare i flussi di dati. SPMD è una sottocategoria di MIMD.

I programmi SPMD di solito hanno la logica necessaria programmata al loro interno per consentire a diversi task di fare una fork o di eseguire condizionatamente solo quelle parti del programma per cui sono state ideate. Cioè, le attività non devono necessariamente eseguire l'intero programma, ma solo una sua parte.

\subsection{Modello MPMD (Multiple Program Multiple Data)} 
\begin{figure}[th]
    \centering
    \includegraphics[width=0.75\linewidth]{img/mpmd.png}
    \caption{Enter Caption}
    \label{fig:enter-label}
\end{figure}
Come SPMD, MPMD (figura \ref{fig:enter-label}) è in realtà un modello di programmazione di "alto livello" che può essere costruito su qualsiasi combinazione dei modelli di programmazione parallela precedentemente menzionati. Applicazioni MPMD hanno tipicamente più file oggetti eseguibili (programmi). Mentre l'applicazione viene eseguita in parallelo, ciascun task può eseguire lo stesso o un altro programa. Tutti i task potrebbero eseguire dati diversi.
	\section{Valutazione delle performance}
La valutazione delle performance può essere fatta su due piani differenti, a seconda dei casi:
\begin{itemize}
    \item il \textbf{tempo di risposta}, anche conosciuto come il \emph{tempo di esecuzione} o \emph{tempo di latenza}, è definito come il tempo necessario per completare un task o un job.

\item il \textbf{throughput} è l'ammontare di lavoro totale eseguito in un dato time slot, a volte chiamato \textbf{larghezza di banda}.
\end{itemize}

Lo speedup è definito come il rapporto tra il tempo di esecuzione di due programmi:
\begin{equation*}
	n=\frac{execTime_y}{execTime_x}
\end{equation*}
di solito $execTime_y$ viene sostituito dal tempo di esecuzione della versione sequenziale di un programma $P$ e $execTime_x$ con la sua versione parallelizzata.

Le misurazioni delle performance possono essere fatte a diversi livelli (considerando solo l'architettura hardware, una sua parte, o una parte di codice, o un intero programma, oppure tutto l'insieme) sfruttando funzionalità contenute in una \textbf{benchmark suite} (un insieme di tool di benchmark, come i \textbf{kernels} che sono piccoli pezzi di codice chiave presi da applicazioni reali, oppure i \textbf{toy programs} che contengono programmi, solitamente lunghi 100 linee di codice, che implementano algoritmi come la moltiplicazione tra matrici, il quicksort, e così via. Vi sono poi i \textbf{benchmark sintetici}, ovvero programmi ideati per simulare il comportamento delle applicazioni reali (come il Linpak, e il Dhrystone). La \textbf{SPEC} (Standard Performance Evaluation Corporation) è un consorzio senza scopo di lucro che stabilisce, mantiene e approva benchmark e strumenti standardizzati per valutare le prestazioni per la nuova generazione di sistemi informatici. SPEC sviluppa suite di benchmark e inoltre esamina e pubblica i risultati presentati dalle nostre organizzazioni membri e da altri licenziatari di benchmark.

Per comparare le performance di un componente hardware può essere consultata una \textbf{tabelle delle performance} dei modelli disponibili in commercio. Le informazioni più importanti riguardano i benchmark per completare un particolare task e il costo del componente hardware.

\subsection{Principio quantitativo} Per aumentare lo speedup è necessario concentrare le energie sul codice che viene eseguito più frequentemente, piuttosto che quello eseguito più raramente.

Un importante riferimento a tal proposito, è la \textbf{legge di Amdahl}
\paragraph{Legge di Amdahl.}
\begin{mdframed}
    \textit{"Il miglioramento delle prestazioni di un sistema che si può ottenere ottimizzando una certa parte del sistema è limitato dalla frazione di tempo in cui tale parte è effettivamente utilizzata"}
\end{mdframed}
Di seguito con $f_i \le 1$ (\textit{"fraction improved"}) viene indicata la frazione del tempo di esecuzione della macchina originale (o del codice originale) che può essere modificato per sfruttare i miglioramenti, mentre con $s_i \ge 1$ (\textit{"speedup improved"}) viene indicato il miglioramento ottenuto da un una modalità di esecuzione più veloce.
\begin{align}
    e_n &= e_o \cdot \left((1-f_i) + \frac{f_i}{s_i} \right) \label{eqn:execution-time-new} \\
    s_g &= \frac{e_o}{e_n}= \frac{1}{(1-f_i)+ \frac{f_i}{s_i}}\label{eqn:speedup-global}
\end{align}
Se un miglioramento può essere usato solo per una frazione dell'intero task:
\begin{equation}
    s_g = \frac{1}{(1-f_i)+\frac{f_i}{s_i}} \fcolorbox{red}{white}{$\le \frac{1}{(1-f_i)}$} \label{eqn:speedup-global-with-limit}
\end{equation}
\begin{eqnarray}
    \begin{tblr}{|c|c|}
    \hline
       e_n & \text{execution time new}
       \\
       \hline
       e_o & \text{execution time old}
       \\
       \hline
       f_i & \text{fraction improved}
       \\
       \hline
       s_i & \text{speedup improved}
       \\
       \hline
       s_g & \text{speedup global}
       \\
       \hline
    \end{tblr}
\end{eqnarray}

Lo \textbf{speedup globale} $s_g$ è uguale a $ \frac{1}{(1-f_i)+\frac{f_i}{s_i}}$ dove $1-f_i$ è la frazione non parallelizzabile, $f_i$ è la frazione parallelizzabile e $s_i$ è lo speedup che si ottiene dalla porzione parallelizzabile di codice. Lo speedup globale massimo, nel riquadro rosso (equazione \ref{eqn:speedup-global-with-limit}) si ottiene facendo tendere la $s_i$ all'infinito: \[\lim_{s_i \to \infty}{\frac{f_i}{s_i}} = 0\]
\begin{exercise}
	Si consideri un miglioramento di 10 volte più veloce della macchina originale (o del codice) ma che può essere applicato solo per il 40\% del tempo. Qual'è il guadagno totale?
\end{exercise}
\begin{solution}
	Traendo i dati dal problema, si ottiene che lo $s_i = 10$ e che $f_i = 40\% = 0,4$. Sostituendo alla formula \ref{eqn:speedup-global} si ottiene
	\begin{equation*}
		s_g = \frac{1}{(1-0.4)+\frac{0.4}{10}} = \frac{1}{0.6+0.04} = \frac{1}{0.64} = \frac{100}{64} = 1.56
	\end{equation*}

	Lo speedup ottenuto è di 1.56.
\end{solution}

\begin{exercise}
	Si consideri una CPU che è stata aggiornata per avere i seguenti cambiamenti:
	\begin{enumerate}
		\item aumentare la velocità di un fattore pari a 5 senza interessare le performance del sistema I/O;
		\item il costo è 5 volte superiore al precedente;
		\item la CPU può essere utilizzata per il 50\% del tempo totale, mentre il rimanente viene impiegato per operazioni di I/O;
		\item il costo della CPU è $\frac{1}{3}$ del costo della macchina.
	\end{enumerate}
	Questo investimento, è conveniente?
\end{exercise}
\begin{solution}
	Lo $s_i = 5$, la $f_i = 50\% = 0.5$. Lo speedup globale è:
	\begin{equation*}
		s_g = \frac{1}{(1-0.5)+\frac{0.5}{5}}=\frac{1}{0.5+0.1} = \frac{10}{6} = 1.67
	\end{equation*}
	il costo è aumentato di:
	\begin{equation*}
		c = 1 \cdot \frac{2}{3} + 5 \cdot \frac{1}{3} =\frac{7}{3} = 2.33
	\end{equation*}
	Dato che il costo è superiore al rendimento ottenuto $ c = 2.33 >  s_g = 1.67 $ non è conveniente fare l'aggiornamento del processore.
\end{solution}

\begin{exercise}
	Si vuole riscrivere un programma su un'architettura MIMD con 100 processori. L'obbiettivo è di ridurre il tempo di esecuzione di 80 volte rispetto a quello precedente su un'architettura SIMD. Qual'è la frazione del programma originale che può restare sequenziale?
\end{exercise}
\begin{solution}
	Dati disponibili: $s_i = 80$. Si sa che $f_i$ è un'incognita.
\end{solution}
    \section{Prospettive sulla programmazione parallela}
Uno degli aspetti più importanti della programmazione parallela, è analizzare il problema e capire se può essere parallelizzato. La parallelizzazione del codice porta a dei miglioramenti delle performance solo se il workload (peso computazionale) è non trascurabile.
Seguono alcune definizioni
\begin{itemize}
    \item \textbf{Task: } unità di esecuzione o di lavoro di un programma o di un sottoprogramma. Le istruzioni sono eseguite sequenzialmente ovvero non in parallelo.

    \item Processo/thread: entità astratta che esegue i task assegnati ai processi. I processi comunicano tra di loro e si sincronizzano per eseguire i loro task.
    \item Processore: motore fisico su cui ciascun processo viene eseguito. Il programma viene scritto come un insieme di processi che vengono poi mappati sul processore.
\end{itemize}

\subsection{Step per creare un programma parallelo}
\begin{figure}[th]
	\centering
	\includegraphics[width=0.7\linewidth]{img/4-step-programma-parallelo}
	\caption{i 4 step per creare un programma parallelo.}
	\label{fig:4-step-programma-parallelo}
\end{figure}

Ci sono 4 passi nella creazione di un programma parallelo:
\begin{enumerate}
    \item Decomposizione dell'algoritmo risolutivo in task o sottoproblemi;
    \item Si assegnano le varie task ai processi; 
    \item Orchestrazione dell'accesso ai dati, comunicazione e sincronizzazione  da parte dei processi;
    \item Mapping dei processi ai processori ()un processo non può essere assegnato a più unità di calcolo).
\end{enumerate}

Il livello di parallelismo viene determinato nelle fasi di decomposizione e assegnamento.

\subsubsection{Comprensione del problema/programma}
Indubbiamente, il primo passo nello sviluppo di software parallelo è capire innanzitutto il problema che si desidera risolvere in parallelo. Se si parte da un programma seriale, ciò implica anche la comprensione del codice esistente.

Prima di investire tempo nel tentativo di sviluppare una soluzione parallela per un problema, bisogna determinare se il problema può effettivamente essere parallelizzato.


\begin{enumerate}
    \item Identificare gli \textbf{hotspot} del programma: rappresentano le parti dove viene svolta la maggior parte del lavoro effettivo. La maggior parte dei programmi scientifici e tecnici di solito svolgono la maggior parte del loro lavoro in pochi punti. Gli strumenti di profilazione e analisi delle prestazioni possono aiutare ad individuare gli hotspote .profiling\footnote{I tool di profiling eseguono il codice e, assieme al suo risultato, ritornano anche un report che può contenere: il numero di invocazioni per ogni funzione contenuta nel codice, e il tempo che impiega ciascuna per determinare il risultato che calcola, quali sono le parti del codice più usate, e così via,\dots} e analisi delle performance (tramite appositi tool). Gli hotspot sono generalmente le sezioni in cui si concentra la parallelizzazione e che hanno un utilizzo della CPU alto.
    \item Identificare i \textbf{bottleneck} del programma: sono le aree di codice che sono più lente da eseguire (come ad esempio le sezioni dedicate all'I/O). Una possibile strada che si può seguire per contrastare i bottleneck, consiste nel trasferire la loro esecuzione sulla GPU, in modo da non dover rallentare la CPU. Dunque, si possono vedere due livelli di parallelismo: il primo dato dalla parallelizzazione su GPU del codice, mentre il secondo dato dalla concorrenza di esecuzione di CPU e GPU.
    \item Identificare gli\textbf{ inibitori al parallelismo}: analizzare il problema significa anche identificare gli inibitori del parallelismo, che sono quei fattori che impediscono di parallelizzare il codice. Una classe comune di inibitori è la dipendenza dei dati, come dimostrato dalla sequenza di Fibonacci sotto.
\end{enumerate}

\underline{Esempio di problema parallelizzabile}: Calcolare l'energia potenziale per migliaia di diverse conformazioni molecolari indipendenti. Una volta completato, individua la conformazione molecolare con l'energia potenziale minima. Ogni conformazione molecolare è determinabile in modo indipendente e il calcolo dell'energia potenziale minima è parallelizzabile.
\\

\underline{Esempio di problema non-parallelizzabile}: Calcolare la serie di Fibonacci. Questo problema non è parallelizzabile in quanto il calcolo attuale dipende dai calcoli precedenti (il termine $k+2$ è dato dalla somma dei termini $k+1$ e $k$). 
\\



\subsubsection{Decomposizione}
Nella decomposizione (o scomposizione) di un problema in task, è cruciale individuarne un numero appropriato in modo che ci siano sempre thread sufficienti per mantenere i processori occupati. In una configurazione con 100 processori funzionanti in parallelo, è necessario avere un ampio pool di task per garantire che tutti i processori siano costantemente utilizzati, anche se uno dei compiti si blocca. È comune che un task raggiunga un punto in cui necessita di dati da un altro task o deve eseguire operazioni di I/O, interrompendo temporaneamente l'esecuzione e lasciando il processore inutilizzato. L'obiettivo è minimizzare questa inattività, assicurando che un nuovo task sia pronto per essere eseguito non appena un processore si libera. Ogni core o processore lasciato inattivo rappresenta uno spreco di risorse. Pertanto, è essenziale disporre di un numero sufficiente di task prontamente disponibili per sostituire quelli bloccati e massimizzare l'utilizzo dei processori.

Esistono due modi per scomporre in task:
\begin{itemize}
    \item \textbf{Scomposizione di dominio:} il dominio può essere suddiviso tra le istanze del task in diverse modalità a seconda delle dimensioni del problema (figura \ref{fig:domain-decomposition}). Ad esempio, se si considera la somma di tutti i dati di un array, questa operazione può essere suddivisa a livello di blocchi o a livello ciclico (figura \ref{fig:slicing-techniques}). Nel primo caso, i dati sono divisi in blocchi distinti assegnati a diverse istanze per l'elaborazione. Nel secondo caso, ogni istanza elabora sequenzialmente una cella di memoria dopo l'altra, seguendo un ciclo, e le celle sono divise da una distanza. 
        \begin{figure}[th]
    	\centering
    	\includegraphics[width=0.7\linewidth]{img/domain-decomposition.png}
    	\caption{scomposizione di dominio.}
    	\label{fig:domain-decomposition}
    \end{figure}
    
    Anche nel caso di domini a due dimensioni, come le matrici, è possibile suddividere l'elaborazione in vari modi. Si possono suddividere i dati in blocchi orizzontali, verticali o a mattonelle. Altrimenti, l'elaborazione può essere ciclica nelle righe, nelle colonne o negli elementi della matrice. Un fattore importante da tenere presente è il bilanciamento del carico (\textbf{load balancing}). se siamo sicuri che il tempo impiegato dalla prima thread sul blocco A corrisponde a quello impiegato dalla seconda thread sul blocco B e così via per il resto dei blocchi, allora abbiamo un carico bilanciato. Questo di solito è il caso per problemi lineari. Quando però abbiamo problemi più complessi, il carico di lavoro di una thread diventa imprevedibile. In questo caso potrebbe essere più ottimale aumentare i blocchi facendo in modo che il primo processore che si libera vada a prendere la successiva thread in attesa.
    \begin{figure}[th]
    	\centering
    	\includegraphics[width=0.7\linewidth]{img/slicing-techniques.png}
    	\caption{i vari modi in cui un dominio può essere scomposto.}
    	\label{fig:slicing-techniques}
    \end{figure}

    \item \textbf{Scomposizione funzionale:} ci si concentra sull'elaborazione che deve essere fatta piuttosto che sui dati manipolati dalla computazione. Il problema è quindi scomposto rispetto al lavoro che deve essere fatto. Ogni task esegue parte del lavoro totale (figura \ref{fig:functional-decomposition}).
        \begin{figure}[th]
    	\centering
    	\includegraphics[width=0.7\linewidth]{img/functional-decomposition.png}
    	\caption{scomposizione funzionale.}
    	\label{fig:functional-decomposition}
    \end{figure}
\end{itemize}

La scomposizione è compito del programmatore, che può essere supportato da tool automatici (campo di ricerca).

\subsubsection*{Esempio di parallelizzazione}
Data un'immagine $N\times N$ vogliamo:
\begin{enumerate}[label=$\bullet$ \textbf{Step \arabic*:}]
    \item raddoppiare la luminosità di ogni pixel; {\color{NavyBlue}\textbf{(complessità $N^2$)}}
    \item calcolare la media di tutti i pixel. {\color{NavyBlue}\textbf{(complessità $N^2$)}}
\end{enumerate}

\paragraph{Soluzione sequenziale.} Una soluzione sequenziale di questo problema costa un tempo totale di {\color{NavyBlue}$N^2+N^2=2N^2$} (figura \ref{fig:esecuzione-sequenziale}). 
\begin{figure}[th]
	\centering
	\includegraphics[width=0.7\linewidth]{img/esecuzione-sequenziale.png}
	\caption{Esecuzione sequenziale.}
	\label{fig:esecuzione-sequenziale}
\end{figure}

\paragraph{Parallelizzazione sulla prima fase.}Una strategia di parallelizzazione consiste nell'eseguire in parallelo lo step 1, completandolo in $\frac{N^2}{P}$ (figure \ref{fig:esecuzione-parallela-1}), dove $P$ rappresenta il numero di processori disponibili. Invece parallelizzare lo step 2 risulta inefficiente  per via della interdipendenza tra i pixel dell'immagine nell'applicazione del filtro, dunque il suo costo è di $N^2$. Lo speedup stimato risulta:
\begin{align*}
    speedup &\le \frac{2N^2}{\frac{N^2}{P} + N^2}\\
    &\le 2
\end{align*}

\begin{figure}[th]
	\centering
	\includegraphics[width=0.7\linewidth]{img/esecuzione-parallela-1.png}
	\caption{Primo tentativo di parallelismo.}
	\label{fig:esecuzione-parallela-1}
\end{figure}

\paragraph{Parallelizzazione sulla seconda fase}Un miglioramento potrebbe essere ottenuto parallelizzando il calcolo della media su ${P}$ processori (calcolo su $P$ matrici di dimensione $\frac{ N^2}{P}$ -- figura \ref{fig:esecuzione-parallela-2}). Infine viene introdotto un tempo di overhead dovuto alla combinazione delle somme parallele per ottenere il risultato finale. I tempi diventano:
\begin{align*}
    \text{Tempo step 1 } &= \frac{N^2}{P}\\
    \text{Tempo step 2 } &= \frac{N^2}{P}+P    
\end{align*}

e si ottiene uno speedup pari a:
\begin{align*}
    speedup \le \frac{2N^2}{\frac{2N^2}{P} + P}\\
\end{align*}

\begin{figure}[th]
	\centering
	\includegraphics[width=0.7\linewidth]{img/esecuzione-parallela-2.png}
	\caption{Secondo tentativo di parallelismo.}
	\label{fig:esecuzione-parallela-2}
\end{figure}

\subsubsection{Assegnamento}
Questo processo richiede la creazione di un numero adeguato di task e la loro distribuzione tra i processi disponibili.

Per prima cosa, è necessario determinare quanti processi sono necessari e quindi associare i task ai processi. Tipicamente, il numero di processi è uguale al numero di processori disponibili per garantire una concorrenza massima.

La strategia di assegnazione dipende dall'esperienza e dalla conoscenza del problema. È consigliabile raggruppare task che richiedono comunicazione tra loro nello stesso processo per accelerare la comunicazione. Al contrario, task che possono eseguire indipendentemente possono essere distribuiti tra processi diversi per ridurre il carico su ciascun processore.

Tuttavia, è cruciale considerare il bilanciamento del carico (\textbf{load balancing}) durante l'assegnazione dei task. Un'assegnazione sbilanciata può compromettere le prestazioni del sistema anche se il codice è parallelizzato correttamente. Ad esempio, se alcuni processi ricevono task più pesanti di altri, potrebbero diventare il collo di bottiglia del sistema, rallentando l'intera esecuzione.

Il tempo di esecuzione complessivo del codice è determinato dal tempo impiegato dal processo più lento, non dalla media dei tempi di esecuzione. Di conseguenza, è fondamentale evitare sbilanciamenti nella distribuzione dei task per massimizzare l'utilizzo dei processori disponibili.

Il miglioramento dell'assegnamento dei task può ridurre gli sprechi di risorse e migliorare le prestazioni complessive del sistema. Anche se non è possibile ottenere un equilibrio perfetto, è importante ridurre al minimo gli sbilanciamenti per sfruttare al meglio le risorse disponibili. Questa ottimizzazione richiede una buona capacità di programmazione e una solida comprensione del problema e dell'architettura del sistema.


\begin{figure}[th]
	\centering
	\includegraphics[width=0.7\linewidth]{img/barrier-sync.png}
	\caption{Esempi di punto di sincronizzazione.}
	\label{fig:barrier-sync}
\end{figure}

Il bilanciamento del carico è cruciale per ottenere prestazioni ottimali in un ambiente di elaborazione parallela, come quello di CUDA. Esistono varie strategie per garantire un bilanciamento efficace del carico di lavoro.

In alcuni casi, è possibile determinare staticamente il peso computazionale di ciascun processo attraverso il profiling e l'analisi delle prestazioni. Tuttavia, ci sono situazioni in cui il peso computazionale dipende dall'input o da altri fattori dinamici, rendendo difficile una determinazione statica. Un esempio di ciò è l'algoritmo di attraversamento del grafo BFS (Breadth-First Search), dove il numero di figli di ciascun nodo può variare notevolmente.

Per gestire questa incertezza, è necessario utilizzare tecniche di bilanciamento dinamico del carico. Queste tecniche consentono di assegnare i task ai processi durante l'esecuzione, in base alle condizioni attuali. Sebbene queste tecniche possano aggiungere un carico computazionale aggiuntivo, sono cruciali per garantire un bilanciamento efficace del carico e massimizzare lo speedup complessivo del sistema.

Durante l'esecuzione, i task vengono inseriti in una coda e assegnati ai processi disponibili. Ogni processo prende un task dalla coda e lo elabora, passando al successivo una volta completato il lavoro. Questo approccio permette di gestire dinamicamente il carico di lavoro e adattarsi alle variazioni nelle prestazioni dei task e dei processi.

Per quanto riguarda la granularità dei task, si parla di task \textbf{coarse-grained} e \textbf{fine-grained}. Questa distinzione si basa sulla quantità di lavoro svolto da ciascun task. I task fine-grained sono piccoli e comprendono poche righe di codice, mentre i task coarse-grained sono più grandi e comprendono una quantità maggiore di codice. La scelta tra task coarse-grained e fine-grained dipende dal problema specifico e dalla necessità di bilanciare il carico di lavoro. I task fine-grained sono più facili da bilanciare e consentono un maggior grado di flessibilità nell'assegnazione dei task ai processi, ma possono introdurre un overhead aggiuntivo a causa delle sincronizzazioni. D'altra parte, i task coarse-grained sono più difficili da bilanciare ma possono ridurre l'overhead di sincronizzazione. La scelta dipende quindi dalle esigenze specifiche del problema e dall'architettura del sistema.
\subsubsection{Orchestrazione}
Il focus principale è comprendere come far comunicare e sincronizzare tutti questi task. Si è già trattato dell'architettura \textit{embarassingly paralle}l, in cui i processi sono completamente indipendenti. In questo scenario, la sincronizzazione non è necessaria: è quasi un'utopia. Tuttavia, quando si rende necessaria la comunicazione, si introduce inevitabilmente un overhead rispetto al codice sequenziale.



\end{document}