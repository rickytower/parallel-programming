\section{Valutazione delle performance}
Il \textbf{tempo di risposta}, anche conosciuto come il \emph{tempo di esecuzione} o \emph{tempo di latenza}, è definito come il tempo necessario per completare un task o un job.

Il \textbf{throughput} è l'ammontare di lavoro totale eseguito in un dato time slot, a volte chiamato \textbf{larghezza di banda}.

Lo speedup è definito come il rapporto tra il tempo di esecuzione di due programmi:
\begin{equation*}
	n=\frac{execTime_y}{execTime_x}
\end{equation*}
di solito $execTime_y$ viene sostituito dal tempo di esecuzione della versione sequenziale di un programma $P$ e $execTime_x$ con la sua versione parallelizzata.