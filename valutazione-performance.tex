\section{Valutazione delle performance}
La valutazione delle performance può essere fatta su due piani differenti, a seconda dei casi:
\begin{itemize}
    \item il \textbf{tempo di risposta}, anche conosciuto come il \emph{tempo di esecuzione} o \emph{tempo di latenza}, è definito come il tempo necessario per completare un task o un job.

\item il \textbf{throughput} è l'ammontare di lavoro totale eseguito in un dato time slot, a volte chiamato \textbf{larghezza di banda}.
\end{itemize}

Lo speedup è definito come il rapporto tra il tempo di esecuzione di due programmi:
\begin{equation*}
	n=\frac{execTime_y}{execTime_x}
\end{equation*}
di solito $execTime_y$ viene sostituito dal tempo di esecuzione della versione sequenziale di un programma $P$ e $execTime_x$ con la sua versione parallelizzata.

Le misurazioni delle performance possono essere fatte a diversi livelli (considerando solo l'architettura hardware, una sua parte, o una parte di codice, o un intero programma, oppure tutto l'insieme) sfruttando funzionalità contenute in una \textbf{benchmark suite} (un insieme di tool di benchmark, come i \textbf{kernels} che sono piccoli pezzi di codice chiave presi da applicazioni reali, oppure i \textbf{toy programs} che contengono programmi, solitamente lunghi 100 linee di codice, che implementano algoritmi come la moltiplicazione tra matrici, il quicksort, e così via. Vi sono poi i \textbf{benchmark sintetici}, ovvero programmi ideati per simulare il comportamento delle applicazioni reali (come il Linpak, e il Dhrystone). La \textbf{SPEC} (Standard Performance Evaluation Corporation) è un consorzio senza scopo di lucro che stabilisce, mantiene e approva benchmark e strumenti standardizzati per valutare le prestazioni per la nuova generazione di sistemi informatici. SPEC sviluppa suite di benchmark e inoltre esamina e pubblica i risultati presentati dalle nostre organizzazioni membri e da altri licenziatari di benchmark.

Per comparare le performance di un componente hardware può essere consultata una \textbf{tabelle delle performance} dei modelli disponibili in commercio. Le informazioni più importanti riguardano i benchmark per completare un particolare task e il costo del componente hardware.

\subsection{Principio quantitativo} Per aumentare lo speedup è necessario concentrare le energie sul codice che viene eseguito più frequentemente, piuttosto che quello eseguito più raramente.

Un importante riferimento a tal proposito, è la \textbf{legge di Amdahl}
\paragraph{Legge di Amdahl.}
\begin{mdframed}
    \textit{"Il miglioramento delle prestazioni di un sistema che si può ottenere ottimizzando una certa parte del sistema è limitato dalla frazione di tempo in cui tale parte è effettivamente utilizzata"}
\end{mdframed}
Di seguito con $f_i \le 1$ (\textit{"fraction improved"}) viene indicata la frazione del tempo di esecuzione della macchina originale (o del codice originale) che può essere modificato per sfruttare i miglioramenti, mentre con $s_i \ge 1$ (\textit{"speedup improved"}) viene indicato il miglioramento ottenuto da un una modalità di esecuzione più veloce.
\begin{align}
    e_n &= e_o \cdot \left((1-f_i) + \frac{f_i}{s_i} \right) \label{eqn:execution-time-new} \\
    s_g &= \frac{e_o}{e_n}- \frac{1}{(1-f_i)+ \frac{f_i}{s_i}}\label{eqn:speedup-global}
\end{align}
Se un miglioramento può essere usato solo per una frazione dell'intero task:
\begin{equation}
    s_g = \frac{1}{(1-f_i)+\frac{f_i}{s_i}} \fcolorbox{red}{white}{$\le \frac{1}{(1-f_i)}$} \label{eqn:speedup-global-with-limit}
\end{equation}
\begin{eqnarray}
    \begin{tblr}{|c|c|}
    \hline
       e_n & \text{execution time new}
       \\
       \hline
       e_o & \text{execution time old}
       \\
       \hline
       f_i & \text{fraction improved}
       \\
       \hline
       s_i & \text{speedup improved}
       \\
       \hline
       s_g & \text{speedup global}
       \\
       \hline
    \end{tblr}
\end{eqnarray}

Lo \textbf{speedup globale} $s_g$ è uguale a $ \frac{1}{(1-f_i)+\frac{f_i}{s_i}}$ dove $1-f_i$ è la frazione non parallelizzabile, $f_i$ è la frazione parallelizzabile e $s_i$ è lo speedup che si ottiene dalla porzione parallelizzabile di codice. Lo speedup globale massimo, nel riquadro rosso (equazione \ref{eqn:speedup-global-with-limit}) si ottiene facendo tendere la $s_i$ all'infinito: \[\lim_{s_i \to \infty}{\frac{f_i}{s_i}} = 0\]

